\documentclass[a4paper,11pt,oneside]{book}
\usepackage[top=1in,bottom=1in,left=1.25in,right=1.25in]{geometry}
\usepackage{CJK}
\usepackage{CJKpunct}
\usepackage[bf,small,center,indentafter,pagestyles]{titlesec}
\usepackage{fancyhdr}
\usepackage{amsmath}
\usepackage{mathtools}
\usepackage[dvips]{graphicx}
\usepackage{subfigure}
\usepackage{algorithm}
\usepackage{algorithmic}

%\newcommand{\song}{\CJKfamily{song}}
%\newcommand{\hei}{\CJKfamily{hei}}
%\newcommand{\kai}{\CJKfamily{kai}}
%\newcommand{\fs}{\CJKfamily{fs}}

%\linespread{1.2}
\pagestyle{plain}
\renewcommand{\baselinestretch}{1.38}
\begin{document}

\title{\textbf{Design and Analysis of Computer Algorithms} \\
\textbf{Homework 1}}
\author{\textbf{Name}: Meng-Ju Lu
\\ \textbf{UTD ID}: 2021323210
\\ \textbf{Email}: mxl162030@utdallas.edu}
\maketitle

\section*{Homework Exercises: Problems 9.3-7, 9.3-8}

1. (Exercise 9.3-7)

Algorithm

(1) Find the median of set $S$ with $SELECT$ algorithm $\Rightarrow O(n)$

(2) Calculate the distances $Distance$ (i.e., $abs(S[i] - median$))$\Rightarrow O(n)$

(3) Find the $K^{th}$ small element $d_k$ in $Distance$ $\Rightarrow O(n)$

(4) Traverse $Distance$, find elements that is smaller than $d_k$ $\Rightarrow O(n)$

Time Complexity: $O(n)$ 

%
\begin{algorithm*}[h]
\caption{\ \ \textbf{SELECT-K($S, k$)}}
\label{selectk}
\begin{algorithmic} [1]
\IF{($ n$ == $k$)}
    	\RETURN S
\ENDIF 
\STATE $median = SELECT($S, 1, n, n/2$)$ 
\STATE $Distance = [], Result = []$
\FOR{($i = 1$ to $n$)}
    \STATE $Distance[i] = abs(S[i] - median)$
\ENDFOR
\STATE $d_k = SELECT(Distance, 1, n, k)$
\FOR{($i = 1$ to $n$)}
    \IF{($ Distance[i] \leq d_k$)}
    	\STATE $Result[i] = S[i]$
    \ENDIF 
\ENDFOR
\RETURN $Result$
\end{algorithmic}
\end{algorithm*}
%

2. (Exercise 9.3-8) Suppose that $X$ and $Y$ are in increasing order. The algorithm used a various of binary search which can be generalized to find the $kth$ element in two sorted arrays.

Algorithm

(1) Get the two median $mx$ and $my$ from $X$ and $Y$. Since $X$, $Y$ are sorted, it costs $O(1)$.

(2) If $mx == my$, then return $mx$. 

(3) If $mx > my$, then then median is present in one of the below two subarrays.

a. From first element of $X$ to $mx$ (X[0...n/2])

b. From $my$ to the last of $Y$ (Y[n/2...n])

(4) If $my > mx$, then the median is present in one of the below two subarrays.

a. From $mx$ to the last of $X$ (X[n/2...n])

b. From first element to $my$ (Y[0...n/2])

Time Complexity: O(lg(n))

%
\begin{algorithm*}[h]
\caption{\ \ \textbf{MEDIAN($X, x_1, x_2, Y, y_1, y_2$)}}
\label{selectk}
\begin{algorithmic} [1]
\STATE $mx = X[(x_1 + x_2) / 2]$  
\STATE $my = Y[(y_1 + y_2) / 2]$
\IF {($mx == my$)}
    \STATE return $mx$
\ENDIF
\IF {($mx < my$)}
    \STATE return $MEDIAN(X, (x_1 + x_2)/2 + 1, x2, Y, y_1, (y_1 + y_2)/2 + 1)$
\ELSE
    \STATE return $MEDIAN(X, x_1, (x_1 + x_2)/2, Y,  (y_1 + y_2)/2 + 1, y_2)$
\ENDIF

\end{algorithmic}
\end{algorithm*}
%

\end{document} 





